\appendix
\label{cha:appendix}

The appendix is an optional part of the manuscript. It can be used to provide additional information that is not essential to the main text. This can include additional tables, figures, or other supplementary material. The appendix is not meant to be a second main part of the manuscript, but rather a complementary part. A typical use case for the appendix are long mathematical derivations or bulky figures (e.g., from empirical experiments) that would interrupt the reading flow of the main text.


\section{First appendix section}
\lipsum[1]

\section{Second appendix section}
Above is a dummy text example using the  \verb|\lipsum| command. This command is useful to generate dummy text for testing the layout of the document. It is part of the \href{https://ctan.org/pkg/lipsum}{lipsum} package. The package is not used in the final version of the manuscript and can be removed before the final version.