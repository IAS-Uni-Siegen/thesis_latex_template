\chapter{Plotting examples using different approaches} 
\label{cha:plotting}

\section{Plotting with tikz}

See the example \figref{fig:tikz_example_01}
\begin{figure}[ht]
    \centering
    \input{fig/tikz/tikz_2Sinus}
    \caption{Example of a tikz figure.}
    \label{fig:tikz_example_01}
\end{figure}
\begin{figure}[ht]
    \centering
    \input{fig/tikz/tikz_Verlauf}
    \caption{Example of another tikz figure.}
    \label{fig:tikz_example_02}
\end{figure}
\begin{figure}[ht]
    \centering
    \input{fig/tikz/tikz_Points}
    \caption{Example of even another tikz figure.}
    \label{fig:tikz_example_03}
\end{figure}



\section{Circuits with circuitikz}
\figref{fig:circuitikz_example} is fully drawn by \LaTeX code. Go to the code to see how this figure is created.
\begin{figure}[ht]
    \centering
    \input{fig/circuitikz/Circ_Widerstand}
    \caption{Example of a circuitikz figure.}
    \label{fig:circuitikz_example}
\end{figure}


\section{Circuits with Inkscape library}
There is a curated Inkscape template for many electrical engineering symbols hosted on Github. \href{https://github.com/upb-lea/Inkscape_electric_Symbols}{https://github.com/upb-lea/Inkscape\_electric\_Symbols}. An overview is shown in \figref{fig:inkscape_example}. The included figure is a pdf-file, what is generated before by an Inkscape drawing.

\begin{figure}[htb]
	\centering
    \includegraphics[]{fig/inkscape/Inkscape_Symbols_All.pdf}
	\caption{Example of an Inkscape figure.}
	\label{fig:inkscape_example}
\end{figure}

